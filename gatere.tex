\documentclass[options]{article}
\begin{document}
\textbf{HOW TO PLAN FOR A POLITICAL COMPAIGN}
\subsection{\textbf{29 March 2017
Prepared by: 
Gatale Elijah (215004651, 15/U/13927/EVE)}}
\section{\textbf{
Introduction}}
A political campaign can be an exciting experience. A great deal will happen between now and the Election with a little forethought and planning. You can be prepared for all the twists and turns and in any cases control the situation. This report is designed to help you anticipate what will happen so as to prepare for it. 
While the given political landscape is an important factor in any campaign, in many cases the most important factor is the difference between winning and losing. 
There are two types of political campaign that have nearly had no chance to achieve victory on Election Day due to their own internal failures.
The first is the campaign that does not have a persuasive language to deliver to voters and also a clear idea of which voters it wants to persuade. This type of campaign lacks direction from the beginning and the situation will only get worse when it does not persuade voters.
Second is the campaign that has a concise, persuasive message and a clear idea of which voters it can persuade but lacks a reasonable plan of what to do between now and Election Day to persuade voters.  
The winning political campaign is most often the one that takes time to target voters, develop a persuasive message and follows through on a reasonable plan to contact the voters.
A political campaign is an intense experience which requires a lot of hard work to be done correctly. There are no tricks or shortcuts to winning the confidence of the voters.
\section{\textbf{	Research}}
It is important to have a complete understanding of a particular situations and the condition in which your campaign will be wedged upon. So by carrying out research, one develops a winning strategy which begins with a realistic assessment of a political landscape in which you will be running for and as part of your research you need to determine the total number of voters, the expected votes to be cast and the number of votes needed to win.
\section{\textbf {Setting a goal}}.
The ultimate goal of every political campaign is winning the elected office. What you need to do here is to determine what must be done to achieve the victory; one way of achieving that is through calculating how many votes will need to guarantee victory and determining where these votes will come from
\section{\textbf{Targeting voters}}
Targeting voters is to determine which subsets of the voting population are most likely to be responsive and then focusing your campaign efforts on these group of voters. This is so because you want to conserve those precious campaign resources of time, money and people and second you want to come up with a clear message that will best persuade those voters you need to convince to vote you.
\section{\textbf{Developing a campaign message}}
Once you have decided who your target audience is, you need to decide what you will say to persuade them to vote for you. This is your campaign message. It tells the voters why you are running for this particular office and why they should choose you over your opponents for the same office. Sounds simple though it is somehow complicated. For example, let us start off by saying what a message is not. A campaign message is not the candidate's program of what they will do if elected, it is not a list of the issues the candidate will address, and it is not a simple, catchy phrase or slogan .All of these things can be part of a campaign message, depending on whether or not they will persuade voters, but they should not be confused with the message, a simple statement that will be repeated over and over throughout the campaign to persuade your target voters. 
\section{\textbf{Conclusion}}
Following the steps abo1ve will make your campaign much more efficient at using your resources of time, money and people. Following these steps will make your campaign much more effective at persuading voters to vote for you. Following these steps will start you well on the road to victory.
\section{\textbf{Reference}}
Politicalscience101.com



\end{document}